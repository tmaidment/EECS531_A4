
% Default to the notebook output style

    


% Inherit from the specified cell style.




    
\documentclass[11pt]{article}

    
    
    \usepackage[T1]{fontenc}
    % Nicer default font (+ math font) than Computer Modern for most use cases
    \usepackage{mathpazo}

    % Basic figure setup, for now with no caption control since it's done
    % automatically by Pandoc (which extracts ![](path) syntax from Markdown).
    \usepackage{graphicx}
    % We will generate all images so they have a width \maxwidth. This means
    % that they will get their normal width if they fit onto the page, but
    % are scaled down if they would overflow the margins.
    \makeatletter
    \def\maxwidth{\ifdim\Gin@nat@width>\linewidth\linewidth
    \else\Gin@nat@width\fi}
    \makeatother
    \let\Oldincludegraphics\includegraphics
    % Set max figure width to be 80% of text width, for now hardcoded.
    \renewcommand{\includegraphics}[1]{\Oldincludegraphics[width=.8\maxwidth]{#1}}
    % Ensure that by default, figures have no caption (until we provide a
    % proper Figure object with a Caption API and a way to capture that
    % in the conversion process - todo).
    \usepackage{caption}
    \DeclareCaptionLabelFormat{nolabel}{}
    \captionsetup{labelformat=nolabel}

    \usepackage{adjustbox} % Used to constrain images to a maximum size 
    \usepackage{xcolor} % Allow colors to be defined
    \usepackage{enumerate} % Needed for markdown enumerations to work
    \usepackage{geometry} % Used to adjust the document margins
    \usepackage{amsmath} % Equations
    \usepackage{amssymb} % Equations
    \usepackage{textcomp} % defines textquotesingle
    % Hack from http://tex.stackexchange.com/a/47451/13684:
    \AtBeginDocument{%
        \def\PYZsq{\textquotesingle}% Upright quotes in Pygmentized code
    }
    \usepackage{upquote} % Upright quotes for verbatim code
    \usepackage{eurosym} % defines \euro
    \usepackage[mathletters]{ucs} % Extended unicode (utf-8) support
    \usepackage[utf8x]{inputenc} % Allow utf-8 characters in the tex document
    \usepackage{fancyvrb} % verbatim replacement that allows latex
    \usepackage{grffile} % extends the file name processing of package graphics 
                         % to support a larger range 
    % The hyperref package gives us a pdf with properly built
    % internal navigation ('pdf bookmarks' for the table of contents,
    % internal cross-reference links, web links for URLs, etc.)
    \usepackage{hyperref}
    \usepackage{longtable} % longtable support required by pandoc >1.10
    \usepackage{booktabs}  % table support for pandoc > 1.12.2
    \usepackage[inline]{enumitem} % IRkernel/repr support (it uses the enumerate* environment)
    \usepackage[normalem]{ulem} % ulem is needed to support strikethroughs (\sout)
                                % normalem makes italics be italics, not underlines
    

    
    
    % Colors for the hyperref package
    \definecolor{urlcolor}{rgb}{0,.145,.698}
    \definecolor{linkcolor}{rgb}{.71,0.21,0.01}
    \definecolor{citecolor}{rgb}{.12,.54,.11}

    % ANSI colors
    \definecolor{ansi-black}{HTML}{3E424D}
    \definecolor{ansi-black-intense}{HTML}{282C36}
    \definecolor{ansi-red}{HTML}{E75C58}
    \definecolor{ansi-red-intense}{HTML}{B22B31}
    \definecolor{ansi-green}{HTML}{00A250}
    \definecolor{ansi-green-intense}{HTML}{007427}
    \definecolor{ansi-yellow}{HTML}{DDB62B}
    \definecolor{ansi-yellow-intense}{HTML}{B27D12}
    \definecolor{ansi-blue}{HTML}{208FFB}
    \definecolor{ansi-blue-intense}{HTML}{0065CA}
    \definecolor{ansi-magenta}{HTML}{D160C4}
    \definecolor{ansi-magenta-intense}{HTML}{A03196}
    \definecolor{ansi-cyan}{HTML}{60C6C8}
    \definecolor{ansi-cyan-intense}{HTML}{258F8F}
    \definecolor{ansi-white}{HTML}{C5C1B4}
    \definecolor{ansi-white-intense}{HTML}{A1A6B2}

    % commands and environments needed by pandoc snippets
    % extracted from the output of `pandoc -s`
    \providecommand{\tightlist}{%
      \setlength{\itemsep}{0pt}\setlength{\parskip}{0pt}}
    \DefineVerbatimEnvironment{Highlighting}{Verbatim}{commandchars=\\\{\}}
    % Add ',fontsize=\small' for more characters per line
    \newenvironment{Shaded}{}{}
    \newcommand{\KeywordTok}[1]{\textcolor[rgb]{0.00,0.44,0.13}{\textbf{{#1}}}}
    \newcommand{\DataTypeTok}[1]{\textcolor[rgb]{0.56,0.13,0.00}{{#1}}}
    \newcommand{\DecValTok}[1]{\textcolor[rgb]{0.25,0.63,0.44}{{#1}}}
    \newcommand{\BaseNTok}[1]{\textcolor[rgb]{0.25,0.63,0.44}{{#1}}}
    \newcommand{\FloatTok}[1]{\textcolor[rgb]{0.25,0.63,0.44}{{#1}}}
    \newcommand{\CharTok}[1]{\textcolor[rgb]{0.25,0.44,0.63}{{#1}}}
    \newcommand{\StringTok}[1]{\textcolor[rgb]{0.25,0.44,0.63}{{#1}}}
    \newcommand{\CommentTok}[1]{\textcolor[rgb]{0.38,0.63,0.69}{\textit{{#1}}}}
    \newcommand{\OtherTok}[1]{\textcolor[rgb]{0.00,0.44,0.13}{{#1}}}
    \newcommand{\AlertTok}[1]{\textcolor[rgb]{1.00,0.00,0.00}{\textbf{{#1}}}}
    \newcommand{\FunctionTok}[1]{\textcolor[rgb]{0.02,0.16,0.49}{{#1}}}
    \newcommand{\RegionMarkerTok}[1]{{#1}}
    \newcommand{\ErrorTok}[1]{\textcolor[rgb]{1.00,0.00,0.00}{\textbf{{#1}}}}
    \newcommand{\NormalTok}[1]{{#1}}
    
    % Additional commands for more recent versions of Pandoc
    \newcommand{\ConstantTok}[1]{\textcolor[rgb]{0.53,0.00,0.00}{{#1}}}
    \newcommand{\SpecialCharTok}[1]{\textcolor[rgb]{0.25,0.44,0.63}{{#1}}}
    \newcommand{\VerbatimStringTok}[1]{\textcolor[rgb]{0.25,0.44,0.63}{{#1}}}
    \newcommand{\SpecialStringTok}[1]{\textcolor[rgb]{0.73,0.40,0.53}{{#1}}}
    \newcommand{\ImportTok}[1]{{#1}}
    \newcommand{\DocumentationTok}[1]{\textcolor[rgb]{0.73,0.13,0.13}{\textit{{#1}}}}
    \newcommand{\AnnotationTok}[1]{\textcolor[rgb]{0.38,0.63,0.69}{\textbf{\textit{{#1}}}}}
    \newcommand{\CommentVarTok}[1]{\textcolor[rgb]{0.38,0.63,0.69}{\textbf{\textit{{#1}}}}}
    \newcommand{\VariableTok}[1]{\textcolor[rgb]{0.10,0.09,0.49}{{#1}}}
    \newcommand{\ControlFlowTok}[1]{\textcolor[rgb]{0.00,0.44,0.13}{\textbf{{#1}}}}
    \newcommand{\OperatorTok}[1]{\textcolor[rgb]{0.40,0.40,0.40}{{#1}}}
    \newcommand{\BuiltInTok}[1]{{#1}}
    \newcommand{\ExtensionTok}[1]{{#1}}
    \newcommand{\PreprocessorTok}[1]{\textcolor[rgb]{0.74,0.48,0.00}{{#1}}}
    \newcommand{\AttributeTok}[1]{\textcolor[rgb]{0.49,0.56,0.16}{{#1}}}
    \newcommand{\InformationTok}[1]{\textcolor[rgb]{0.38,0.63,0.69}{\textbf{\textit{{#1}}}}}
    \newcommand{\WarningTok}[1]{\textcolor[rgb]{0.38,0.63,0.69}{\textbf{\textit{{#1}}}}}
    
    
    % Define a nice break command that doesn't care if a line doesn't already
    % exist.
    \def\br{\hspace*{\fill} \\* }
    % Math Jax compatability definitions
    \def\gt{>}
    \def\lt{<}
    % Document parameters
    \title{EECS531 - A4 - E2 - tdm47}
    
    
    

    % Pygments definitions
    
\makeatletter
\def\PY@reset{\let\PY@it=\relax \let\PY@bf=\relax%
    \let\PY@ul=\relax \let\PY@tc=\relax%
    \let\PY@bc=\relax \let\PY@ff=\relax}
\def\PY@tok#1{\csname PY@tok@#1\endcsname}
\def\PY@toks#1+{\ifx\relax#1\empty\else%
    \PY@tok{#1}\expandafter\PY@toks\fi}
\def\PY@do#1{\PY@bc{\PY@tc{\PY@ul{%
    \PY@it{\PY@bf{\PY@ff{#1}}}}}}}
\def\PY#1#2{\PY@reset\PY@toks#1+\relax+\PY@do{#2}}

\expandafter\def\csname PY@tok@w\endcsname{\def\PY@tc##1{\textcolor[rgb]{0.73,0.73,0.73}{##1}}}
\expandafter\def\csname PY@tok@c\endcsname{\let\PY@it=\textit\def\PY@tc##1{\textcolor[rgb]{0.25,0.50,0.50}{##1}}}
\expandafter\def\csname PY@tok@cp\endcsname{\def\PY@tc##1{\textcolor[rgb]{0.74,0.48,0.00}{##1}}}
\expandafter\def\csname PY@tok@k\endcsname{\let\PY@bf=\textbf\def\PY@tc##1{\textcolor[rgb]{0.00,0.50,0.00}{##1}}}
\expandafter\def\csname PY@tok@kp\endcsname{\def\PY@tc##1{\textcolor[rgb]{0.00,0.50,0.00}{##1}}}
\expandafter\def\csname PY@tok@kt\endcsname{\def\PY@tc##1{\textcolor[rgb]{0.69,0.00,0.25}{##1}}}
\expandafter\def\csname PY@tok@o\endcsname{\def\PY@tc##1{\textcolor[rgb]{0.40,0.40,0.40}{##1}}}
\expandafter\def\csname PY@tok@ow\endcsname{\let\PY@bf=\textbf\def\PY@tc##1{\textcolor[rgb]{0.67,0.13,1.00}{##1}}}
\expandafter\def\csname PY@tok@nb\endcsname{\def\PY@tc##1{\textcolor[rgb]{0.00,0.50,0.00}{##1}}}
\expandafter\def\csname PY@tok@nf\endcsname{\def\PY@tc##1{\textcolor[rgb]{0.00,0.00,1.00}{##1}}}
\expandafter\def\csname PY@tok@nc\endcsname{\let\PY@bf=\textbf\def\PY@tc##1{\textcolor[rgb]{0.00,0.00,1.00}{##1}}}
\expandafter\def\csname PY@tok@nn\endcsname{\let\PY@bf=\textbf\def\PY@tc##1{\textcolor[rgb]{0.00,0.00,1.00}{##1}}}
\expandafter\def\csname PY@tok@ne\endcsname{\let\PY@bf=\textbf\def\PY@tc##1{\textcolor[rgb]{0.82,0.25,0.23}{##1}}}
\expandafter\def\csname PY@tok@nv\endcsname{\def\PY@tc##1{\textcolor[rgb]{0.10,0.09,0.49}{##1}}}
\expandafter\def\csname PY@tok@no\endcsname{\def\PY@tc##1{\textcolor[rgb]{0.53,0.00,0.00}{##1}}}
\expandafter\def\csname PY@tok@nl\endcsname{\def\PY@tc##1{\textcolor[rgb]{0.63,0.63,0.00}{##1}}}
\expandafter\def\csname PY@tok@ni\endcsname{\let\PY@bf=\textbf\def\PY@tc##1{\textcolor[rgb]{0.60,0.60,0.60}{##1}}}
\expandafter\def\csname PY@tok@na\endcsname{\def\PY@tc##1{\textcolor[rgb]{0.49,0.56,0.16}{##1}}}
\expandafter\def\csname PY@tok@nt\endcsname{\let\PY@bf=\textbf\def\PY@tc##1{\textcolor[rgb]{0.00,0.50,0.00}{##1}}}
\expandafter\def\csname PY@tok@nd\endcsname{\def\PY@tc##1{\textcolor[rgb]{0.67,0.13,1.00}{##1}}}
\expandafter\def\csname PY@tok@s\endcsname{\def\PY@tc##1{\textcolor[rgb]{0.73,0.13,0.13}{##1}}}
\expandafter\def\csname PY@tok@sd\endcsname{\let\PY@it=\textit\def\PY@tc##1{\textcolor[rgb]{0.73,0.13,0.13}{##1}}}
\expandafter\def\csname PY@tok@si\endcsname{\let\PY@bf=\textbf\def\PY@tc##1{\textcolor[rgb]{0.73,0.40,0.53}{##1}}}
\expandafter\def\csname PY@tok@se\endcsname{\let\PY@bf=\textbf\def\PY@tc##1{\textcolor[rgb]{0.73,0.40,0.13}{##1}}}
\expandafter\def\csname PY@tok@sr\endcsname{\def\PY@tc##1{\textcolor[rgb]{0.73,0.40,0.53}{##1}}}
\expandafter\def\csname PY@tok@ss\endcsname{\def\PY@tc##1{\textcolor[rgb]{0.10,0.09,0.49}{##1}}}
\expandafter\def\csname PY@tok@sx\endcsname{\def\PY@tc##1{\textcolor[rgb]{0.00,0.50,0.00}{##1}}}
\expandafter\def\csname PY@tok@m\endcsname{\def\PY@tc##1{\textcolor[rgb]{0.40,0.40,0.40}{##1}}}
\expandafter\def\csname PY@tok@gh\endcsname{\let\PY@bf=\textbf\def\PY@tc##1{\textcolor[rgb]{0.00,0.00,0.50}{##1}}}
\expandafter\def\csname PY@tok@gu\endcsname{\let\PY@bf=\textbf\def\PY@tc##1{\textcolor[rgb]{0.50,0.00,0.50}{##1}}}
\expandafter\def\csname PY@tok@gd\endcsname{\def\PY@tc##1{\textcolor[rgb]{0.63,0.00,0.00}{##1}}}
\expandafter\def\csname PY@tok@gi\endcsname{\def\PY@tc##1{\textcolor[rgb]{0.00,0.63,0.00}{##1}}}
\expandafter\def\csname PY@tok@gr\endcsname{\def\PY@tc##1{\textcolor[rgb]{1.00,0.00,0.00}{##1}}}
\expandafter\def\csname PY@tok@ge\endcsname{\let\PY@it=\textit}
\expandafter\def\csname PY@tok@gs\endcsname{\let\PY@bf=\textbf}
\expandafter\def\csname PY@tok@gp\endcsname{\let\PY@bf=\textbf\def\PY@tc##1{\textcolor[rgb]{0.00,0.00,0.50}{##1}}}
\expandafter\def\csname PY@tok@go\endcsname{\def\PY@tc##1{\textcolor[rgb]{0.53,0.53,0.53}{##1}}}
\expandafter\def\csname PY@tok@gt\endcsname{\def\PY@tc##1{\textcolor[rgb]{0.00,0.27,0.87}{##1}}}
\expandafter\def\csname PY@tok@err\endcsname{\def\PY@bc##1{\setlength{\fboxsep}{0pt}\fcolorbox[rgb]{1.00,0.00,0.00}{1,1,1}{\strut ##1}}}
\expandafter\def\csname PY@tok@kc\endcsname{\let\PY@bf=\textbf\def\PY@tc##1{\textcolor[rgb]{0.00,0.50,0.00}{##1}}}
\expandafter\def\csname PY@tok@kd\endcsname{\let\PY@bf=\textbf\def\PY@tc##1{\textcolor[rgb]{0.00,0.50,0.00}{##1}}}
\expandafter\def\csname PY@tok@kn\endcsname{\let\PY@bf=\textbf\def\PY@tc##1{\textcolor[rgb]{0.00,0.50,0.00}{##1}}}
\expandafter\def\csname PY@tok@kr\endcsname{\let\PY@bf=\textbf\def\PY@tc##1{\textcolor[rgb]{0.00,0.50,0.00}{##1}}}
\expandafter\def\csname PY@tok@bp\endcsname{\def\PY@tc##1{\textcolor[rgb]{0.00,0.50,0.00}{##1}}}
\expandafter\def\csname PY@tok@fm\endcsname{\def\PY@tc##1{\textcolor[rgb]{0.00,0.00,1.00}{##1}}}
\expandafter\def\csname PY@tok@vc\endcsname{\def\PY@tc##1{\textcolor[rgb]{0.10,0.09,0.49}{##1}}}
\expandafter\def\csname PY@tok@vg\endcsname{\def\PY@tc##1{\textcolor[rgb]{0.10,0.09,0.49}{##1}}}
\expandafter\def\csname PY@tok@vi\endcsname{\def\PY@tc##1{\textcolor[rgb]{0.10,0.09,0.49}{##1}}}
\expandafter\def\csname PY@tok@vm\endcsname{\def\PY@tc##1{\textcolor[rgb]{0.10,0.09,0.49}{##1}}}
\expandafter\def\csname PY@tok@sa\endcsname{\def\PY@tc##1{\textcolor[rgb]{0.73,0.13,0.13}{##1}}}
\expandafter\def\csname PY@tok@sb\endcsname{\def\PY@tc##1{\textcolor[rgb]{0.73,0.13,0.13}{##1}}}
\expandafter\def\csname PY@tok@sc\endcsname{\def\PY@tc##1{\textcolor[rgb]{0.73,0.13,0.13}{##1}}}
\expandafter\def\csname PY@tok@dl\endcsname{\def\PY@tc##1{\textcolor[rgb]{0.73,0.13,0.13}{##1}}}
\expandafter\def\csname PY@tok@s2\endcsname{\def\PY@tc##1{\textcolor[rgb]{0.73,0.13,0.13}{##1}}}
\expandafter\def\csname PY@tok@sh\endcsname{\def\PY@tc##1{\textcolor[rgb]{0.73,0.13,0.13}{##1}}}
\expandafter\def\csname PY@tok@s1\endcsname{\def\PY@tc##1{\textcolor[rgb]{0.73,0.13,0.13}{##1}}}
\expandafter\def\csname PY@tok@mb\endcsname{\def\PY@tc##1{\textcolor[rgb]{0.40,0.40,0.40}{##1}}}
\expandafter\def\csname PY@tok@mf\endcsname{\def\PY@tc##1{\textcolor[rgb]{0.40,0.40,0.40}{##1}}}
\expandafter\def\csname PY@tok@mh\endcsname{\def\PY@tc##1{\textcolor[rgb]{0.40,0.40,0.40}{##1}}}
\expandafter\def\csname PY@tok@mi\endcsname{\def\PY@tc##1{\textcolor[rgb]{0.40,0.40,0.40}{##1}}}
\expandafter\def\csname PY@tok@il\endcsname{\def\PY@tc##1{\textcolor[rgb]{0.40,0.40,0.40}{##1}}}
\expandafter\def\csname PY@tok@mo\endcsname{\def\PY@tc##1{\textcolor[rgb]{0.40,0.40,0.40}{##1}}}
\expandafter\def\csname PY@tok@ch\endcsname{\let\PY@it=\textit\def\PY@tc##1{\textcolor[rgb]{0.25,0.50,0.50}{##1}}}
\expandafter\def\csname PY@tok@cm\endcsname{\let\PY@it=\textit\def\PY@tc##1{\textcolor[rgb]{0.25,0.50,0.50}{##1}}}
\expandafter\def\csname PY@tok@cpf\endcsname{\let\PY@it=\textit\def\PY@tc##1{\textcolor[rgb]{0.25,0.50,0.50}{##1}}}
\expandafter\def\csname PY@tok@c1\endcsname{\let\PY@it=\textit\def\PY@tc##1{\textcolor[rgb]{0.25,0.50,0.50}{##1}}}
\expandafter\def\csname PY@tok@cs\endcsname{\let\PY@it=\textit\def\PY@tc##1{\textcolor[rgb]{0.25,0.50,0.50}{##1}}}

\def\PYZbs{\char`\\}
\def\PYZus{\char`\_}
\def\PYZob{\char`\{}
\def\PYZcb{\char`\}}
\def\PYZca{\char`\^}
\def\PYZam{\char`\&}
\def\PYZlt{\char`\<}
\def\PYZgt{\char`\>}
\def\PYZsh{\char`\#}
\def\PYZpc{\char`\%}
\def\PYZdl{\char`\$}
\def\PYZhy{\char`\-}
\def\PYZsq{\char`\'}
\def\PYZdq{\char`\"}
\def\PYZti{\char`\~}
% for compatibility with earlier versions
\def\PYZat{@}
\def\PYZlb{[}
\def\PYZrb{]}
\makeatother


    % Exact colors from NB
    \definecolor{incolor}{rgb}{0.0, 0.0, 0.5}
    \definecolor{outcolor}{rgb}{0.545, 0.0, 0.0}



    
    % Prevent overflowing lines due to hard-to-break entities
    \sloppy 
    % Setup hyperref package
    \hypersetup{
      breaklinks=true,  % so long urls are correctly broken across lines
      colorlinks=true,
      urlcolor=urlcolor,
      linkcolor=linkcolor,
      citecolor=citecolor,
      }
    % Slightly bigger margins than the latex defaults
    
    \geometry{verbose,tmargin=1in,bmargin=1in,lmargin=1in,rmargin=1in}
    
    

    \begin{document}
    
    
    \maketitle
    
    

    
    \hypertarget{eecs-531---e2---a4}{%
\subsection{EECS 531 - E2 - A4}\label{eecs-531---e2---a4}}

\hypertarget{tristan-maidment-tdm47}{%
\subsubsection{Tristan Maidment (tdm47)}\label{tristan-maidment-tdm47}}

\hypertarget{goal}{%
\paragraph{Goal}\label{goal}}

The purpose of this exercise is to implement the Lucas Kanade optical
flow algorithm.

\hypertarget{implementation}{%
\paragraph{Implementation}\label{implementation}}

    \begin{Verbatim}[commandchars=\\\{\}]
{\color{incolor}In [{\color{incolor}1}]:} \PY{o}{\PYZpc{}}\PY{k}{matplotlib} inline
        \PY{k+kn}{import} \PY{n+nn}{numpy} \PY{k}{as} \PY{n+nn}{np}
        \PY{k+kn}{import} \PY{n+nn}{cv2}
        \PY{k+kn}{import} \PY{n+nn}{matplotlib}\PY{n+nn}{.}\PY{n+nn}{pyplot} \PY{k}{as} \PY{n+nn}{plt}
        \PY{k+kn}{from} \PY{n+nn}{pims} \PY{k}{import} \PY{n}{ImageSequence}
        \PY{k+kn}{from} \PY{n+nn}{numpy}\PY{n+nn}{.}\PY{n+nn}{linalg} \PY{k}{import} \PY{n}{lstsq}
        \PY{k+kn}{import} \PY{n+nn}{timeit}
\end{Verbatim}


    To test the optical flow algorithm, we will be testing it on two image
sequences.

    \begin{Verbatim}[commandchars=\\\{\}]
{\color{incolor}In [{\color{incolor}2}]:} \PY{n}{img\PYZus{}rubic} \PY{o}{=} \PY{n}{ImageSequence}\PY{p}{(}\PY{l+s+s1}{\PYZsq{}}\PY{l+s+s1}{./data/rubic/rubic.*.png}\PY{l+s+s1}{\PYZsq{}}\PY{p}{,} \PY{n}{as\PYZus{}grey}\PY{o}{=}\PY{k+kc}{True}\PY{p}{)}
        \PY{n}{img\PYZus{}sphere} \PY{o}{=} \PY{n}{ImageSequence}\PY{p}{(}\PY{l+s+s1}{\PYZsq{}}\PY{l+s+s1}{./data/sphere/sphere.*.png}\PY{l+s+s1}{\PYZsq{}}\PY{p}{,} \PY{n}{as\PYZus{}grey}\PY{o}{=}\PY{k+kc}{True}\PY{p}{)}
\end{Verbatim}


    As part of the exercise, I will be doing the following sub-question in
order.

\textbf{a)} Compute the discrete derivative components in the motion
gradient constraint equation for a single pixel location of a moving
image, \texttt{I(x,y,t)}.

During this step, the spatial gradient \texttt{I\_x,\ I\_y} is
calculated by taking the difference of the neighboring pixels. In
addition, we estimate temporal gradient, \texttt{I\_t} by taking the
intensity difference between the two time slices.

    \begin{Verbatim}[commandchars=\\\{\}]
{\color{incolor}In [{\color{incolor}3}]:} \PY{k}{def} \PY{n+nf}{gradient\PYZus{}components}\PY{p}{(}\PY{n}{img0}\PY{p}{,} \PY{n}{img1}\PY{p}{,} \PY{n}{x}\PY{p}{,} \PY{n}{y}\PY{p}{)}\PY{p}{:}
            \PY{n}{height}\PY{p}{,} \PY{n}{width} \PY{o}{=} \PY{n}{img0}\PY{o}{.}\PY{n}{shape}
            \PY{n}{I\PYZus{}x} \PY{o}{=} \PY{n}{np}\PY{o}{.}\PY{n}{subtract}\PY{p}{(}\PY{n}{img0}\PY{p}{[}\PY{n}{x}\PY{o}{+}\PY{l+m+mi}{1}\PY{p}{,} \PY{n}{y}\PY{p}{]}\PY{p}{,}\PY{n}{img0}\PY{p}{[}\PY{n}{x}\PY{o}{\PYZhy{}}\PY{l+m+mi}{1}\PY{p}{,}\PY{n}{y}\PY{p}{]}\PY{p}{)}\PY{o}{/}\PY{l+m+mi}{2} \PY{k}{if} \PY{n}{x} \PY{o}{\PYZgt{}} \PY{l+m+mi}{0} \PY{o+ow}{and} \PY{n}{x} \PY{o}{\PYZlt{}} \PY{n}{width} \PY{k}{else} \PY{l+m+mi}{0}
            \PY{n}{I\PYZus{}y} \PY{o}{=} \PY{n}{np}\PY{o}{.}\PY{n}{subtract}\PY{p}{(}\PY{n}{img0}\PY{p}{[}\PY{n}{x}\PY{p}{,} \PY{n}{y}\PY{o}{+}\PY{l+m+mi}{1}\PY{p}{]}\PY{p}{,}\PY{n}{img0}\PY{p}{[}\PY{n}{x}\PY{p}{,}\PY{n}{y}\PY{o}{\PYZhy{}}\PY{l+m+mi}{1}\PY{p}{]}\PY{p}{)}\PY{o}{/}\PY{l+m+mi}{2} \PY{k}{if} \PY{n}{y} \PY{o}{\PYZgt{}} \PY{l+m+mi}{0} \PY{o+ow}{and} \PY{n}{y} \PY{o}{\PYZlt{}} \PY{n}{height} \PY{k}{else} \PY{l+m+mi}{0}
            \PY{n}{I\PYZus{}t} \PY{o}{=} \PY{n}{np}\PY{o}{.}\PY{n}{subtract}\PY{p}{(}\PY{n}{img1}\PY{p}{[}\PY{n}{x}\PY{p}{,}\PY{n}{y}\PY{p}{]}\PY{p}{,} \PY{n}{img0}\PY{p}{[}\PY{n}{x}\PY{p}{,}\PY{n}{y}\PY{p}{]}\PY{p}{)} \PY{k}{if} \PY{n}{y} \PY{o}{\PYZgt{}} \PY{l+m+mi}{0} \PY{o+ow}{and} \PY{n}{y} \PY{o}{\PYZlt{}} \PY{n}{height} \PY{o+ow}{and} \PY{n}{x} \PY{o}{\PYZgt{}} \PY{l+m+mi}{0} \PY{o+ow}{and} \PY{n}{x} \PY{o}{\PYZlt{}} \PY{n}{width} \PY{k}{else} \PY{l+m+mi}{0} \PY{c+c1}{\PYZsh{} I\PYZus{}t approximator}
            \PY{k}{return} \PY{n}{I\PYZus{}x}\PY{p}{,} \PY{n}{I\PYZus{}y}\PY{p}{,} \PY{n}{I\PYZus{}t}
\end{Verbatim}


    \begin{Verbatim}[commandchars=\\\{\}]
{\color{incolor}In [{\color{incolor}4}]:} \PY{n}{I\PYZus{}x}\PY{p}{,} \PY{n}{I\PYZus{}y}\PY{p}{,} \PY{n}{I\PYZus{}t} \PY{o}{=} \PY{n}{gradient\PYZus{}components}\PY{p}{(}\PY{n}{img\PYZus{}rubic}\PY{p}{[}\PY{l+m+mi}{0}\PY{p}{]}\PY{o}{/}\PY{l+m+mi}{255}\PY{p}{,} \PY{n}{img\PYZus{}rubic}\PY{p}{[}\PY{l+m+mi}{1}\PY{p}{]}\PY{o}{/}\PY{l+m+mi}{255}\PY{p}{,} \PY{l+m+mi}{100}\PY{p}{,} \PY{l+m+mi}{100}\PY{p}{)}
        \PY{n+nb}{print}\PY{p}{(}\PY{l+s+s2}{\PYZdq{}}\PY{l+s+s2}{I\PYZus{}x:}\PY{l+s+s2}{\PYZdq{}}\PY{p}{,} \PY{n}{I\PYZus{}x}\PY{p}{,} \PY{l+s+s2}{\PYZdq{}}\PY{l+s+s2}{I\PYZus{}y:}\PY{l+s+s2}{\PYZdq{}}\PY{p}{,} \PY{n}{I\PYZus{}y}\PY{p}{,} \PY{l+s+s2}{\PYZdq{}}\PY{l+s+s2}{I\PYZus{}t:}\PY{l+s+s2}{\PYZdq{}}\PY{p}{,} \PY{n}{I\PYZus{}t}\PY{p}{)}
\end{Verbatim}


    \begin{Verbatim}[commandchars=\\\{\}]
I\_x: 0.15490196078431373 I\_y: -0.0215686274509804 I\_t: -0.02745098039215685

    \end{Verbatim}

    \textbf{b)} Compute the matrix \textbf{A} and vector \textbf{b}
representing the motion constraint equations over a (small) \emph{n × n}
window centered on pixel \texttt{(x,\ y)}.

Matrix \textbf{A} is of size \texttt{n\^{}2\ *\ 2}. The left column
contains a vectorized column of the \emph{x} gradients and the right
column contains a vectorized column of the \emph{y} gradients.

Matrix \textbf{b} contains a vectorized list of the temporal gradients.

    \begin{Verbatim}[commandchars=\\\{\}]
{\color{incolor}In [{\color{incolor}5}]:} \PY{c+c1}{\PYZsh{} n = window size}
        \PY{c+c1}{\PYZsh{} I will be using a gradient estimator kernel,}
        \PY{c+c1}{\PYZsh{} originally used for edge detection to calculate A and B}
        \PY{k}{def} \PY{n+nf}{compute\PYZus{}Ab}\PY{p}{(}\PY{n}{img0}\PY{p}{,} \PY{n}{img1}\PY{p}{,} \PY{n}{x}\PY{p}{,} \PY{n}{y}\PY{p}{,} \PY{n}{n}\PY{p}{)}\PY{p}{:}
            \PY{n}{A} \PY{o}{=} \PY{n}{np}\PY{o}{.}\PY{n}{zeros}\PY{p}{(}\PY{p}{(}\PY{n}{n}\PY{o}{*}\PY{o}{*}\PY{l+m+mi}{2}\PY{p}{,} \PY{l+m+mi}{2}\PY{p}{)}\PY{p}{)}
            \PY{n}{b} \PY{o}{=} \PY{n}{np}\PY{o}{.}\PY{n}{zeros}\PY{p}{(}\PY{n}{n}\PY{o}{*}\PY{o}{*}\PY{l+m+mi}{2}\PY{p}{)}
            \PY{n}{offset} \PY{o}{=} \PY{n}{np}\PY{o}{.}\PY{n}{floor}\PY{p}{(}\PY{n}{n}\PY{o}{/}\PY{l+m+mi}{2}\PY{p}{)}\PY{p}{;} 
            
            \PY{k}{for} \PY{n}{j} \PY{o+ow}{in} \PY{n}{np}\PY{o}{.}\PY{n}{arange}\PY{p}{(}\PY{n}{n}\PY{p}{)}\PY{p}{:}
                \PY{k}{for} \PY{n}{i} \PY{o+ow}{in} \PY{n}{np}\PY{o}{.}\PY{n}{arange}\PY{p}{(}\PY{n}{n}\PY{p}{)}\PY{p}{:}
                    \PY{n}{I\PYZus{}x}\PY{p}{,} \PY{n}{I\PYZus{}y}\PY{p}{,} \PY{n}{I\PYZus{}t} \PY{o}{=} \PY{n}{gradient\PYZus{}components}\PY{p}{(}\PY{n}{img0}\PY{p}{,} \PY{n}{img1}\PY{p}{,} \PY{n+nb}{int}\PY{p}{(}\PY{n}{x} \PY{o}{+} \PY{n}{i} \PY{o}{\PYZhy{}} \PY{n}{offset}\PY{p}{)}\PY{p}{,} \PY{n+nb}{int}\PY{p}{(}\PY{n}{y} \PY{o}{+} \PY{n}{j} \PY{o}{\PYZhy{}} \PY{n}{offset}\PY{p}{)}\PY{p}{)}
                    \PY{n}{A}\PY{p}{[}\PY{p}{(}\PY{n}{i}\PY{o}{*}\PY{n}{n}\PY{p}{)} \PY{o}{+} \PY{n}{j}\PY{p}{,}\PY{l+m+mi}{0}\PY{p}{]} \PY{o}{=} \PY{n}{I\PYZus{}x}
                    \PY{n}{A}\PY{p}{[}\PY{p}{(}\PY{n}{i}\PY{o}{*}\PY{n}{n}\PY{p}{)} \PY{o}{+} \PY{n}{j}\PY{p}{,}\PY{l+m+mi}{1}\PY{p}{]} \PY{o}{=} \PY{n}{I\PYZus{}y}
                    \PY{n}{b}\PY{p}{[}\PY{p}{(}\PY{n}{i}\PY{o}{*}\PY{n}{n}\PY{p}{)} \PY{o}{+} \PY{n}{j}\PY{p}{]} \PY{o}{=} \PY{n}{I\PYZus{}t}
                
            \PY{k}{return} \PY{n}{A}\PY{p}{,} \PY{n}{b}
\end{Verbatim}


    \begin{Verbatim}[commandchars=\\\{\}]
{\color{incolor}In [{\color{incolor}6}]:} \PY{c+c1}{\PYZsh{}print(compute\PYZus{}Ab(img\PYZus{}rubic[0]/255, img\PYZus{}rubic[1]/255, 100, 100, 5))}
\end{Verbatim}


    \textbf{c)} Solve the constraint equations either directly using the
formula in the lecture slides or using a least- squares solver to
estimate motion (u, v) at point (x, y).

Now that we have a vectorized 2D linear representation of the of the
data surrounding the point in question, the \textbf{d} matrix
\texttt{(u,\ v)} contains the motion vector at that point. We have the
equation in the form \texttt{A\ *\ d\ =\ b}

This can be found via least squares regression.

    \begin{Verbatim}[commandchars=\\\{\}]
{\color{incolor}In [{\color{incolor}7}]:} \PY{k}{def} \PY{n+nf}{compute\PYZus{}uv}\PY{p}{(}\PY{n}{A}\PY{p}{,} \PY{n}{b}\PY{p}{)}\PY{p}{:}
            \PY{c+c1}{\PYZsh{}print(np.matmul(A.T, A).shape, np.matmul(A.T, \PYZhy{}b).shape)}
            \PY{c+c1}{\PYZsh{}uv = linregress(np.matmul(A.T, A).shape, np.matmul(A.T, \PYZhy{}b).shape)}
            \PY{n}{uv} \PY{o}{=} \PY{n}{lstsq}\PY{p}{(}\PY{n}{np}\PY{o}{.}\PY{n}{matmul}\PY{p}{(}\PY{n}{A}\PY{o}{.}\PY{n}{T}\PY{p}{,} \PY{n}{A}\PY{p}{)}\PY{p}{,} \PY{o}{\PYZhy{}}\PY{n}{np}\PY{o}{.}\PY{n}{matmul}\PY{p}{(}\PY{n}{A}\PY{o}{.}\PY{n}{T}\PY{p}{,} \PY{n}{b}\PY{p}{)}\PY{p}{,} \PY{n}{rcond}\PY{o}{=}\PY{k+kc}{None}\PY{p}{)}
            \PY{k}{return} \PY{n}{uv}\PY{p}{[}\PY{l+m+mi}{0}\PY{p}{]}
\end{Verbatim}


    \begin{Verbatim}[commandchars=\\\{\}]
{\color{incolor}In [{\color{incolor}8}]:} \PY{n}{A}\PY{p}{,}\PY{n}{b} \PY{o}{=} \PY{n}{compute\PYZus{}Ab}\PY{p}{(}\PY{n}{img\PYZus{}sphere}\PY{p}{[}\PY{l+m+mi}{0}\PY{p}{]}\PY{o}{/}\PY{l+m+mi}{255}\PY{p}{,} \PY{n}{img\PYZus{}sphere}\PY{p}{[}\PY{l+m+mi}{1}\PY{p}{]}\PY{o}{/}\PY{l+m+mi}{255}\PY{p}{,} \PY{l+m+mi}{100}\PY{p}{,} \PY{l+m+mi}{100}\PY{p}{,} \PY{l+m+mi}{5}\PY{p}{)}
        \PY{n}{uv} \PY{o}{=} \PY{n}{compute\PYZus{}uv}\PY{p}{(}\PY{n}{A}\PY{p}{,} \PY{n}{b}\PY{p}{)}
        \PY{n+nb}{print}\PY{p}{(}\PY{n}{uv}\PY{p}{)}
\end{Verbatim}


    \begin{Verbatim}[commandchars=\\\{\}]
[0.30035571 2.27552987]

    \end{Verbatim}

    \textbf{d)} Estimate the motion for an grid of locations covering the
image.

Now that we have the ability to get the motion vector at any point, we
can simply test the direction of movement at various points around the
picture. Using \emph{matplotlib's} \texttt{quiver} functionality, we can
create a nice representation of the optical flow.

    \begin{Verbatim}[commandchars=\\\{\}]
{\color{incolor}In [{\color{incolor}9}]:} \PY{k}{def} \PY{n+nf}{optical\PYZus{}flow\PYZus{}field}\PY{p}{(}\PY{n}{img0}\PY{p}{,} \PY{n}{img1}\PY{p}{,} \PY{n}{window\PYZus{}size}\PY{p}{,} \PY{n}{grid\PYZus{}size}\PY{p}{)}\PY{p}{:}
            \PY{n}{height}\PY{p}{,} \PY{n}{width} \PY{o}{=} \PY{n}{img1}\PY{o}{.}\PY{n}{shape}
            
            \PY{n}{grid\PYZus{}x}\PY{p}{,} \PY{n}{grid\PYZus{}y} \PY{o}{=} \PY{n}{grid\PYZus{}size}\PY{p}{[}\PY{l+m+mi}{0}\PY{p}{]}\PY{p}{,} \PY{n}{grid\PYZus{}size}\PY{p}{[}\PY{l+m+mi}{1}\PY{p}{]}
            
            \PY{n}{flow} \PY{o}{=} \PY{n}{np}\PY{o}{.}\PY{n}{zeros}\PY{p}{(}\PY{p}{(}\PY{n}{grid\PYZus{}x}\PY{p}{,} \PY{n}{grid\PYZus{}y}\PY{p}{,} \PY{l+m+mi}{2}\PY{p}{)}\PY{p}{)}
            \PY{n}{delta\PYZus{}x}\PY{p}{,} \PY{n}{delta\PYZus{}y} \PY{o}{=} \PY{n+nb}{int}\PY{p}{(}\PY{n}{height}\PY{o}{/}\PY{n}{grid\PYZus{}x}\PY{p}{)}\PY{p}{,} \PY{n+nb}{int}\PY{p}{(}\PY{n}{width}\PY{o}{/}\PY{n}{grid\PYZus{}y}\PY{p}{)}
            
            \PY{n}{x} \PY{o}{=} \PY{n}{np}\PY{o}{.}\PY{n}{arange}\PY{p}{(}\PY{l+m+mi}{0}\PY{p}{,} \PY{n}{width}\PY{p}{,} \PY{n}{delta\PYZus{}x}\PY{p}{)} \PY{o}{+} \PY{n}{np}\PY{o}{.}\PY{n}{floor}\PY{p}{(}\PY{n}{delta\PYZus{}x}\PY{o}{/}\PY{l+m+mi}{2}\PY{p}{)}
            \PY{n}{y} \PY{o}{=} \PY{n}{np}\PY{o}{.}\PY{n}{arange}\PY{p}{(}\PY{l+m+mi}{0}\PY{p}{,} \PY{n}{height}\PY{p}{,} \PY{n}{delta\PYZus{}y}\PY{p}{)} \PY{o}{+} \PY{n}{np}\PY{o}{.}\PY{n}{floor}\PY{p}{(}\PY{n}{delta\PYZus{}y}\PY{o}{/}\PY{l+m+mi}{2}\PY{p}{)}
            \PY{n}{gridX}\PY{p}{,} \PY{n}{gridY} \PY{o}{=} \PY{n}{np}\PY{o}{.}\PY{n}{meshgrid}\PY{p}{(}\PY{n}{x}\PY{p}{,}\PY{n}{y}\PY{p}{)}\PY{p}{;}
            
            \PY{k}{for} \PY{n}{x} \PY{o+ow}{in} \PY{n}{np}\PY{o}{.}\PY{n}{arange}\PY{p}{(}\PY{l+m+mi}{0}\PY{p}{,} \PY{n}{grid\PYZus{}x}\PY{p}{)}\PY{p}{:}
                \PY{k}{for} \PY{n}{y} \PY{o+ow}{in} \PY{n}{np}\PY{o}{.}\PY{n}{arange}\PY{p}{(}\PY{l+m+mi}{0}\PY{p}{,} \PY{n}{grid\PYZus{}y}\PY{p}{)}\PY{p}{:}
                    \PY{n}{A}\PY{p}{,}\PY{n}{b} \PY{o}{=} \PY{n}{compute\PYZus{}Ab}\PY{p}{(}\PY{n}{img0}\PY{p}{,} \PY{n}{img1}\PY{p}{,} \PY{n}{x}\PY{o}{*}\PY{n}{delta\PYZus{}x}\PY{p}{,} \PY{n}{y}\PY{o}{*}\PY{n}{delta\PYZus{}y}\PY{p}{,} \PY{n}{window\PYZus{}size}\PY{p}{)}
                    \PY{n}{uv} \PY{o}{=} \PY{n}{compute\PYZus{}uv}\PY{p}{(}\PY{n}{A}\PY{p}{,} \PY{n}{b}\PY{p}{)}
                    \PY{n}{flow}\PY{p}{[}\PY{n}{x}\PY{p}{,}\PY{n}{y}\PY{p}{,}\PY{p}{:}\PY{p}{]} \PY{o}{=} \PY{p}{(}\PY{n}{uv}\PY{p}{[}\PY{l+m+mi}{0}\PY{p}{]}\PY{p}{,} \PY{n}{uv}\PY{p}{[}\PY{l+m+mi}{1}\PY{p}{]}\PY{p}{)}
            \PY{k}{return} \PY{p}{(}\PY{n}{gridX}\PY{p}{,} \PY{n}{gridY}\PY{p}{)}\PY{p}{,} \PY{n}{flow}
\end{Verbatim}


    \begin{Verbatim}[commandchars=\\\{\}]
{\color{incolor}In [{\color{incolor}10}]:} \PY{n}{grid}\PY{p}{,} \PY{n}{flow} \PY{o}{=} \PY{n}{optical\PYZus{}flow\PYZus{}field}\PY{p}{(}\PY{n}{img\PYZus{}rubic}\PY{p}{[}\PY{l+m+mi}{0}\PY{p}{]}\PY{o}{/}\PY{l+m+mi}{255}\PY{p}{,} \PY{n}{img\PYZus{}rubic}\PY{p}{[}\PY{l+m+mi}{1}\PY{p}{]}\PY{o}{/}\PY{l+m+mi}{255}\PY{p}{,} \PY{l+m+mi}{14}\PY{p}{,} \PY{p}{(}\PY{l+m+mi}{30}\PY{p}{,} \PY{l+m+mi}{32}\PY{p}{)}\PY{p}{)}
\end{Verbatim}


    \begin{Verbatim}[commandchars=\\\{\}]
{\color{incolor}In [{\color{incolor}11}]:} \PY{n}{fig} \PY{o}{=} \PY{n}{plt}\PY{o}{.}\PY{n}{figure}\PY{p}{(}\PY{n}{figsize}\PY{o}{=}\PY{p}{(}\PY{l+m+mi}{10}\PY{p}{,} \PY{l+m+mi}{10}\PY{p}{)}\PY{p}{,} \PY{n}{dpi}\PY{o}{=}\PY{l+m+mi}{80}\PY{p}{)}
         \PY{n}{plt}\PY{o}{.}\PY{n}{imshow}\PY{p}{(}\PY{n}{img\PYZus{}rubic}\PY{p}{[}\PY{l+m+mi}{1}\PY{p}{]}\PY{p}{,} \PY{n}{cmap}\PY{o}{=}\PY{l+s+s1}{\PYZsq{}}\PY{l+s+s1}{gray}\PY{l+s+s1}{\PYZsq{}}\PY{p}{)}
         \PY{n}{plt}\PY{o}{.}\PY{n}{quiver}\PY{p}{(}\PY{n}{grid}\PY{p}{[}\PY{l+m+mi}{0}\PY{p}{]}\PY{p}{,} \PY{n}{grid}\PY{p}{[}\PY{l+m+mi}{1}\PY{p}{]}\PY{p}{,} \PY{n}{flow}\PY{p}{[}\PY{p}{:}\PY{p}{,}\PY{p}{:}\PY{p}{,}\PY{l+m+mi}{1}\PY{p}{]}\PY{p}{,} \PY{n}{flow}\PY{p}{[}\PY{p}{:}\PY{p}{,}\PY{p}{:}\PY{p}{,}\PY{l+m+mi}{0}\PY{p}{]}\PY{p}{,} \PY{n}{color}\PY{o}{=}\PY{l+s+s1}{\PYZsq{}}\PY{l+s+s1}{lime}\PY{l+s+s1}{\PYZsq{}}\PY{p}{,} \PY{n}{angles}\PY{o}{=}\PY{l+s+s2}{\PYZdq{}}\PY{l+s+s2}{xy}\PY{l+s+s2}{\PYZdq{}}\PY{p}{,}\PY{n}{scale}\PY{o}{=}\PY{l+m+mi}{40}\PY{p}{)}
         \PY{n}{plt}\PY{o}{.}\PY{n}{show}\PY{p}{(}\PY{p}{)}
\end{Verbatim}


    \begin{center}
    \adjustimage{max size={0.9\linewidth}{0.9\paperheight}}{output_16_0.png}
    \end{center}
    { \hspace*{\fill} \\}
    
    \hypertarget{optimization}{%
\paragraph{Optimization}\label{optimization}}

Due to the order of the steps required by the directions, the
calculation of the optical flow via the Lucas-Kanade algorithm is
greatly unoptimzied. For example, the gradient is at each coordinate
point is calculated \texttt{(n\^{}2)-1} more times that it needs to be.
For this reason, I have implemented an optimized for of this algorithm
that avoids that problem, by only calculating the gradient once.

Another optimzation that will be used is the use of gradient estimators.
To estimate the gradient we will be using the \emph{sobel kernel}, which
I used for edge detection earlier this year.

In addition, I added the ability to use the singular values of matrix
\textbf{A} as the coefficient for the motion vector. This helps with
optical flow for certain image sequences, but significantly decreases
the optical flow detection for other sequences.

    \begin{Verbatim}[commandchars=\\\{\}]
{\color{incolor}In [{\color{incolor}12}]:} \PY{k}{def} \PY{n+nf}{lucas\PYZus{}kanade\PYZus{}optimzed}\PY{p}{(}\PY{n}{img0}\PY{p}{,} \PY{n}{img1}\PY{p}{,} \PY{n}{window\PYZus{}size}\PY{p}{,} \PY{n}{grid\PYZus{}size}\PY{p}{,} \PY{n}{svd}\PY{o}{=}\PY{k+kc}{False}\PY{p}{)}\PY{p}{:}
             \PY{n}{height}\PY{p}{,} \PY{n}{width} \PY{o}{=} \PY{n}{img0}\PY{o}{.}\PY{n}{shape}
             
             \PY{c+c1}{\PYZsh{}sobel kernels}
             \PY{n}{kernel\PYZus{}x} \PY{o}{=} \PY{n}{np}\PY{o}{.}\PY{n}{array}\PY{p}{(}\PY{p}{[}\PY{p}{[}\PY{o}{\PYZhy{}}\PY{l+m+mi}{1}\PY{p}{,} \PY{l+m+mi}{0}\PY{p}{,} \PY{l+m+mi}{1}\PY{p}{]}\PY{p}{,} \PY{p}{[}\PY{o}{\PYZhy{}}\PY{l+m+mi}{2}\PY{p}{,} \PY{l+m+mi}{0}\PY{p}{,} \PY{l+m+mi}{2}\PY{p}{]}\PY{p}{,} \PY{p}{[}\PY{o}{\PYZhy{}}\PY{l+m+mi}{1}\PY{p}{,} \PY{l+m+mi}{0}\PY{p}{,} \PY{l+m+mi}{1}\PY{p}{]}\PY{p}{]}\PY{p}{)} \PY{o}{*} \PY{l+m+mi}{1}\PY{o}{/}\PY{l+m+mi}{8}
             \PY{n}{kernel\PYZus{}y} \PY{o}{=} \PY{n}{np}\PY{o}{.}\PY{n}{array}\PY{p}{(}\PY{p}{[}\PY{p}{[}\PY{o}{\PYZhy{}}\PY{l+m+mi}{1}\PY{p}{,} \PY{o}{\PYZhy{}}\PY{l+m+mi}{2}\PY{p}{,} \PY{o}{\PYZhy{}}\PY{l+m+mi}{1}\PY{p}{]}\PY{p}{,} \PY{p}{[}\PY{l+m+mi}{0}\PY{p}{,} \PY{l+m+mi}{0}\PY{p}{,} \PY{l+m+mi}{0}\PY{p}{]}\PY{p}{,} \PY{p}{[}\PY{l+m+mi}{1}\PY{p}{,} \PY{l+m+mi}{2}\PY{p}{,} \PY{l+m+mi}{1}\PY{p}{]}\PY{p}{]}\PY{p}{)} \PY{o}{*} \PY{l+m+mi}{1}\PY{o}{/}\PY{l+m+mi}{8}
             
             \PY{n}{grad\PYZus{}y} \PY{o}{=} \PY{n}{cv2}\PY{o}{.}\PY{n}{filter2D}\PY{p}{(}\PY{n}{img0}\PY{p}{,} \PY{o}{\PYZhy{}}\PY{l+m+mi}{1}\PY{p}{,} \PY{n}{kernel\PYZus{}x}\PY{p}{)}
             \PY{n}{grad\PYZus{}x} \PY{o}{=} \PY{n}{cv2}\PY{o}{.}\PY{n}{filter2D}\PY{p}{(}\PY{n}{img0}\PY{p}{,} \PY{o}{\PYZhy{}}\PY{l+m+mi}{1}\PY{p}{,} \PY{n}{kernel\PYZus{}y}\PY{p}{)}
             \PY{n}{grad\PYZus{}z} \PY{o}{=} \PY{n}{np}\PY{o}{.}\PY{n}{subtract}\PY{p}{(}\PY{n}{img1}\PY{p}{,} \PY{n}{img0}\PY{p}{)}
             
             \PY{n}{grid\PYZus{}x}\PY{p}{,} \PY{n}{grid\PYZus{}y} \PY{o}{=} \PY{n}{grid\PYZus{}size}\PY{p}{[}\PY{l+m+mi}{0}\PY{p}{]}\PY{p}{,} \PY{n}{grid\PYZus{}size}\PY{p}{[}\PY{l+m+mi}{1}\PY{p}{]}
             
             \PY{n}{flow} \PY{o}{=} \PY{n}{np}\PY{o}{.}\PY{n}{zeros}\PY{p}{(}\PY{p}{(}\PY{n}{grid\PYZus{}x}\PY{p}{,} \PY{n}{grid\PYZus{}y}\PY{p}{,} \PY{l+m+mi}{2}\PY{p}{)}\PY{p}{)}
             \PY{n}{delta\PYZus{}x}\PY{p}{,} \PY{n}{delta\PYZus{}y} \PY{o}{=} \PY{n+nb}{int}\PY{p}{(}\PY{n}{height}\PY{o}{/}\PY{n}{grid\PYZus{}x}\PY{p}{)}\PY{p}{,} \PY{n+nb}{int}\PY{p}{(}\PY{n}{width}\PY{o}{/}\PY{n}{grid\PYZus{}y}\PY{p}{)}
             
             \PY{n}{x} \PY{o}{=} \PY{n}{np}\PY{o}{.}\PY{n}{arange}\PY{p}{(}\PY{l+m+mi}{0}\PY{p}{,} \PY{n}{width}\PY{p}{,} \PY{n}{delta\PYZus{}x}\PY{p}{)} \PY{o}{+} \PY{n}{np}\PY{o}{.}\PY{n}{floor}\PY{p}{(}\PY{n}{delta\PYZus{}x}\PY{o}{/}\PY{l+m+mi}{2}\PY{p}{)}
             \PY{n}{y} \PY{o}{=} \PY{n}{np}\PY{o}{.}\PY{n}{arange}\PY{p}{(}\PY{l+m+mi}{0}\PY{p}{,} \PY{n}{height}\PY{p}{,} \PY{n}{delta\PYZus{}y}\PY{p}{)} \PY{o}{+} \PY{n}{np}\PY{o}{.}\PY{n}{floor}\PY{p}{(}\PY{n}{delta\PYZus{}y}\PY{o}{/}\PY{l+m+mi}{2}\PY{p}{)}
             \PY{n}{gridX}\PY{p}{,} \PY{n}{gridY} \PY{o}{=} \PY{n}{np}\PY{o}{.}\PY{n}{meshgrid}\PY{p}{(}\PY{n}{x}\PY{p}{,}\PY{n}{y}\PY{p}{)}\PY{p}{;}
             
             \PY{k}{for} \PY{n}{i} \PY{o+ow}{in} \PY{n}{np}\PY{o}{.}\PY{n}{arange}\PY{p}{(}\PY{l+m+mi}{0}\PY{p}{,} \PY{n}{grid\PYZus{}x}\PY{p}{)}\PY{p}{:}
                 \PY{k}{for} \PY{n}{j} \PY{o+ow}{in} \PY{n}{np}\PY{o}{.}\PY{n}{arange}\PY{p}{(}\PY{l+m+mi}{0}\PY{p}{,} \PY{n}{grid\PYZus{}y}\PY{p}{)}\PY{p}{:}
                     \PY{n}{uv} \PY{o}{=} \PY{n}{eval\PYZus{}region}\PY{p}{(}\PY{n}{grad\PYZus{}x}\PY{p}{,} \PY{n}{grad\PYZus{}y}\PY{p}{,} \PY{n}{grad\PYZus{}z}\PY{p}{,} \PY{n}{i}\PY{o}{*}\PY{n}{delta\PYZus{}x}\PY{p}{,} \PY{n}{j}\PY{o}{*}\PY{n}{delta\PYZus{}y}\PY{p}{,} \PY{n}{window\PYZus{}size}\PY{p}{,} \PY{n}{svd}\PY{p}{)}
                     \PY{n}{flow}\PY{p}{[}\PY{n}{i}\PY{p}{,}\PY{n}{j}\PY{p}{,}\PY{p}{:}\PY{p}{]} \PY{o}{=} \PY{p}{(}\PY{n}{uv}\PY{p}{[}\PY{l+m+mi}{0}\PY{p}{]}\PY{p}{,} \PY{n}{uv}\PY{p}{[}\PY{l+m+mi}{1}\PY{p}{]}\PY{p}{)}
             \PY{k}{return} \PY{p}{(}\PY{n}{gridX}\PY{p}{,} \PY{n}{gridY}\PY{p}{)}\PY{p}{,} \PY{n}{flow}
             
         \PY{k}{def} \PY{n+nf}{eval\PYZus{}region}\PY{p}{(}\PY{n}{grad\PYZus{}x}\PY{p}{,} \PY{n}{grad\PYZus{}y}\PY{p}{,} \PY{n}{grad\PYZus{}z}\PY{p}{,} \PY{n}{x}\PY{p}{,} \PY{n}{y}\PY{p}{,} \PY{n}{n}\PY{p}{,} \PY{n}{svd}\PY{p}{)}\PY{p}{:} 
             \PY{n}{A} \PY{o}{=} \PY{n}{np}\PY{o}{.}\PY{n}{zeros}\PY{p}{(}\PY{p}{(}\PY{n}{n}\PY{o}{*}\PY{o}{*}\PY{l+m+mi}{2}\PY{p}{,} \PY{l+m+mi}{2}\PY{p}{)}\PY{p}{)}
             \PY{n}{b} \PY{o}{=} \PY{n}{np}\PY{o}{.}\PY{n}{zeros}\PY{p}{(}\PY{n}{n}\PY{o}{*}\PY{o}{*}\PY{l+m+mi}{2}\PY{p}{)}
             \PY{n}{offset} \PY{o}{=} \PY{n}{np}\PY{o}{.}\PY{n}{floor}\PY{p}{(}\PY{n}{n}\PY{o}{/}\PY{l+m+mi}{2}\PY{p}{)}
             \PY{n}{x\PYZus{}} \PY{o}{=} \PY{n+nb}{int}\PY{p}{(}\PY{n}{x} \PY{o}{\PYZhy{}} \PY{n}{offset}\PY{p}{)}
             \PY{n}{y\PYZus{}} \PY{o}{=} \PY{n+nb}{int}\PY{p}{(}\PY{n}{y} \PY{o}{\PYZhy{}} \PY{n}{offset}\PY{p}{)}
             
             \PY{k}{for} \PY{n}{j} \PY{o+ow}{in} \PY{n}{np}\PY{o}{.}\PY{n}{arange}\PY{p}{(}\PY{n}{n}\PY{p}{)}\PY{p}{:}
                 \PY{k}{for} \PY{n}{i} \PY{o+ow}{in} \PY{n}{np}\PY{o}{.}\PY{n}{arange}\PY{p}{(}\PY{n}{n}\PY{p}{)}\PY{p}{:}
                     \PY{n}{A}\PY{p}{[}\PY{p}{(}\PY{n}{i}\PY{o}{*}\PY{n}{n}\PY{p}{)} \PY{o}{+} \PY{n}{j}\PY{p}{,}\PY{l+m+mi}{0}\PY{p}{]} \PY{o}{=} \PY{n}{grad\PYZus{}x}\PY{p}{[}\PY{n}{x\PYZus{}} \PY{o}{+} \PY{n}{i}\PY{p}{,} \PY{n}{y\PYZus{}} \PY{o}{+} \PY{n}{j}\PY{p}{]}
                     \PY{n}{A}\PY{p}{[}\PY{p}{(}\PY{n}{i}\PY{o}{*}\PY{n}{n}\PY{p}{)} \PY{o}{+} \PY{n}{j}\PY{p}{,}\PY{l+m+mi}{1}\PY{p}{]} \PY{o}{=} \PY{n}{grad\PYZus{}y}\PY{p}{[}\PY{n}{x\PYZus{}} \PY{o}{+} \PY{n}{i}\PY{p}{,} \PY{n}{y\PYZus{}} \PY{o}{+} \PY{n}{j}\PY{p}{]}
                     \PY{n}{b}\PY{p}{[}\PY{p}{(}\PY{n}{i}\PY{o}{*}\PY{n}{n}\PY{p}{)} \PY{o}{+} \PY{n}{j}\PY{p}{]} \PY{o}{=} \PY{n}{grad\PYZus{}z}\PY{p}{[}\PY{n}{x\PYZus{}} \PY{o}{+} \PY{n}{i}\PY{p}{,} \PY{n}{y\PYZus{}} \PY{o}{+} \PY{n}{j}\PY{p}{]}
             \PY{n}{uv} \PY{o}{=} \PY{n}{lstsq}\PY{p}{(}\PY{n}{np}\PY{o}{.}\PY{n}{matmul}\PY{p}{(}\PY{n}{A}\PY{o}{.}\PY{n}{T}\PY{p}{,} \PY{n}{A}\PY{p}{)}\PY{p}{,} \PY{o}{\PYZhy{}}\PY{n}{np}\PY{o}{.}\PY{n}{matmul}\PY{p}{(}\PY{n}{A}\PY{o}{.}\PY{n}{T}\PY{p}{,} \PY{n}{b}\PY{p}{)}\PY{p}{,} \PY{n}{rcond}\PY{o}{=}\PY{k+kc}{None}\PY{p}{)}
             \PY{k}{return} \PY{n}{np}\PY{o}{.}\PY{n}{multiply}\PY{p}{(}\PY{n}{uv}\PY{p}{[}\PY{l+m+mi}{0}\PY{p}{]}\PY{p}{,} \PY{n}{uv}\PY{p}{[}\PY{l+m+mi}{3}\PY{p}{]}\PY{p}{)} \PY{k}{if} \PY{n}{svd} \PY{k}{else} \PY{n}{uv}\PY{p}{[}\PY{l+m+mi}{0}\PY{p}{]}
\end{Verbatim}


    \hypertarget{timing-comparison}{%
\paragraph{Timing Comparison}\label{timing-comparison}}

Using the ore optimized method for this method is significantly faster
when processing the optical flow of the images.

    \begin{Verbatim}[commandchars=\\\{\}]
{\color{incolor}In [{\color{incolor}13}]:} \PY{o}{\PYZpc{}}\PY{k}{timeit} optical\PYZus{}flow\PYZus{}field(img\PYZus{}rubic[0]/255, img\PYZus{}rubic[1]/255, 14, (30, 32))
         \PY{o}{\PYZpc{}}\PY{k}{timeit} lucas\PYZus{}kanade\PYZus{}optimzed(img\PYZus{}rubic[0]/255, img\PYZus{}rubic[1]/255, 14, (30, 32), False)
\end{Verbatim}


    \begin{Verbatim}[commandchars=\\\{\}]
3.02 s ± 199 ms per loop (mean ± std. dev. of 7 runs, 1 loop each)
626 ms ± 19.4 ms per loop (mean ± std. dev. of 7 runs, 1 loop each)

    \end{Verbatim}

    \begin{Verbatim}[commandchars=\\\{\}]
{\color{incolor}In [{\color{incolor}14}]:} \PY{n}{grid}\PY{p}{,} \PY{n}{flow} \PY{o}{=} \PY{n}{lucas\PYZus{}kanade\PYZus{}optimzed}\PY{p}{(}\PY{n}{img\PYZus{}rubic}\PY{p}{[}\PY{l+m+mi}{0}\PY{p}{]}\PY{o}{/}\PY{l+m+mi}{255}\PY{p}{,} \PY{n}{img\PYZus{}rubic}\PY{p}{[}\PY{l+m+mi}{1}\PY{p}{]}\PY{o}{/}\PY{l+m+mi}{255}\PY{p}{,} \PY{l+m+mi}{14}\PY{p}{,} \PY{p}{(}\PY{l+m+mi}{30}\PY{p}{,} \PY{l+m+mi}{32}\PY{p}{)}\PY{p}{)}
         \PY{n}{fig} \PY{o}{=} \PY{n}{plt}\PY{o}{.}\PY{n}{figure}\PY{p}{(}\PY{n}{figsize}\PY{o}{=}\PY{p}{(}\PY{l+m+mi}{10}\PY{p}{,} \PY{l+m+mi}{10}\PY{p}{)}\PY{p}{,} \PY{n}{dpi}\PY{o}{=}\PY{l+m+mi}{80}\PY{p}{)}
         \PY{n}{plt}\PY{o}{.}\PY{n}{imshow}\PY{p}{(}\PY{n}{img\PYZus{}rubic}\PY{p}{[}\PY{l+m+mi}{1}\PY{p}{]}\PY{p}{,} \PY{n}{cmap}\PY{o}{=}\PY{l+s+s1}{\PYZsq{}}\PY{l+s+s1}{gray}\PY{l+s+s1}{\PYZsq{}}\PY{p}{)}
         \PY{n}{plt}\PY{o}{.}\PY{n}{quiver}\PY{p}{(}\PY{n}{grid}\PY{p}{[}\PY{l+m+mi}{0}\PY{p}{]}\PY{p}{,} \PY{n}{grid}\PY{p}{[}\PY{l+m+mi}{1}\PY{p}{]}\PY{p}{,} \PY{n}{flow}\PY{p}{[}\PY{p}{:}\PY{p}{,}\PY{p}{:}\PY{p}{,}\PY{l+m+mi}{1}\PY{p}{]}\PY{p}{,} \PY{n}{flow}\PY{p}{[}\PY{p}{:}\PY{p}{,}\PY{p}{:}\PY{p}{,}\PY{l+m+mi}{0}\PY{p}{]}\PY{p}{,} \PY{n}{color}\PY{o}{=}\PY{l+s+s1}{\PYZsq{}}\PY{l+s+s1}{lime}\PY{l+s+s1}{\PYZsq{}}\PY{p}{,} \PY{n}{angles}\PY{o}{=}\PY{l+s+s2}{\PYZdq{}}\PY{l+s+s2}{xy}\PY{l+s+s2}{\PYZdq{}}\PY{p}{,}\PY{n}{scale}\PY{o}{=}\PY{l+m+mi}{50}\PY{p}{)}
         \PY{n}{plt}\PY{o}{.}\PY{n}{show}\PY{p}{(}\PY{p}{)}
\end{Verbatim}


    \begin{center}
    \adjustimage{max size={0.9\linewidth}{0.9\paperheight}}{output_21_0.png}
    \end{center}
    { \hspace*{\fill} \\}
    
    \begin{Verbatim}[commandchars=\\\{\}]
{\color{incolor}In [{\color{incolor}15}]:} \PY{n}{grid}\PY{p}{,} \PY{n}{flow} \PY{o}{=} \PY{n}{lucas\PYZus{}kanade\PYZus{}optimzed}\PY{p}{(}\PY{n}{img\PYZus{}sphere}\PY{p}{[}\PY{l+m+mi}{0}\PY{p}{]}\PY{o}{/}\PY{l+m+mi}{255}\PY{p}{,} \PY{n}{img\PYZus{}sphere}\PY{p}{[}\PY{l+m+mi}{1}\PY{p}{]}\PY{o}{/}\PY{l+m+mi}{255}\PY{p}{,} \PY{l+m+mi}{8}\PY{p}{,} \PY{p}{(}\PY{l+m+mi}{40}\PY{p}{,} \PY{l+m+mi}{40}\PY{p}{)}\PY{p}{)}
         \PY{n}{fig} \PY{o}{=} \PY{n}{plt}\PY{o}{.}\PY{n}{figure}\PY{p}{(}\PY{n}{figsize}\PY{o}{=}\PY{p}{(}\PY{l+m+mi}{10}\PY{p}{,} \PY{l+m+mi}{10}\PY{p}{)}\PY{p}{,} \PY{n}{dpi}\PY{o}{=}\PY{l+m+mi}{80}\PY{p}{)}
         \PY{n}{plt}\PY{o}{.}\PY{n}{imshow}\PY{p}{(}\PY{n}{img\PYZus{}sphere}\PY{p}{[}\PY{l+m+mi}{1}\PY{p}{]}\PY{p}{,} \PY{n}{cmap}\PY{o}{=}\PY{l+s+s1}{\PYZsq{}}\PY{l+s+s1}{gray}\PY{l+s+s1}{\PYZsq{}}\PY{p}{)}
         \PY{n}{plt}\PY{o}{.}\PY{n}{quiver}\PY{p}{(}\PY{n}{grid}\PY{p}{[}\PY{l+m+mi}{0}\PY{p}{]}\PY{p}{,} \PY{n}{grid}\PY{p}{[}\PY{l+m+mi}{1}\PY{p}{]}\PY{p}{,} \PY{n}{flow}\PY{p}{[}\PY{p}{:}\PY{p}{,}\PY{p}{:}\PY{p}{,}\PY{l+m+mi}{1}\PY{p}{]}\PY{p}{,} \PY{n}{flow}\PY{p}{[}\PY{p}{:}\PY{p}{,}\PY{p}{:}\PY{p}{,}\PY{l+m+mi}{0}\PY{p}{]}\PY{p}{,} \PY{n}{color}\PY{o}{=}\PY{l+s+s1}{\PYZsq{}}\PY{l+s+s1}{lime}\PY{l+s+s1}{\PYZsq{}}\PY{p}{,} \PY{n}{angles}\PY{o}{=}\PY{l+s+s2}{\PYZdq{}}\PY{l+s+s2}{xy}\PY{l+s+s2}{\PYZdq{}}\PY{p}{,}\PY{n}{scale}\PY{o}{=}\PY{l+m+mi}{50}\PY{p}{)}
         \PY{n}{plt}\PY{o}{.}\PY{n}{show}\PY{p}{(}\PY{p}{)}
\end{Verbatim}


    \begin{center}
    \adjustimage{max size={0.9\linewidth}{0.9\paperheight}}{output_22_0.png}
    \end{center}
    { \hspace*{\fill} \\}
    
    \hypertarget{singular-values}{%
\paragraph{Singular Values}\label{singular-values}}

By using the Singular Values of matrix \textbf{A} as a coefficient for
the motion vectors, we can get significantly better detection for the
Rubic's cube. This is due to the fact that the noisy sections (small
changes in intensity) will inherently have lower singular values.

However, due to the lack of noise and small intensity value changes
created by the translating gradient patches of the sphere, the singular
values are very small along the surface, resulting in incorrectly
truncated motion vectors.

    \begin{Verbatim}[commandchars=\\\{\}]
{\color{incolor}In [{\color{incolor}16}]:} \PY{n}{grid}\PY{p}{,} \PY{n}{flow} \PY{o}{=} \PY{n}{lucas\PYZus{}kanade\PYZus{}optimzed}\PY{p}{(}\PY{n}{img\PYZus{}rubic}\PY{p}{[}\PY{l+m+mi}{0}\PY{p}{]}\PY{o}{/}\PY{l+m+mi}{255}\PY{p}{,} \PY{n}{img\PYZus{}rubic}\PY{p}{[}\PY{l+m+mi}{1}\PY{p}{]}\PY{o}{/}\PY{l+m+mi}{255}\PY{p}{,} \PY{l+m+mi}{14}\PY{p}{,} \PY{p}{(}\PY{l+m+mi}{30}\PY{p}{,} \PY{l+m+mi}{32}\PY{p}{)}\PY{p}{,} \PY{k+kc}{True}\PY{p}{)}
         \PY{n}{fig} \PY{o}{=} \PY{n}{plt}\PY{o}{.}\PY{n}{figure}\PY{p}{(}\PY{n}{figsize}\PY{o}{=}\PY{p}{(}\PY{l+m+mi}{10}\PY{p}{,} \PY{l+m+mi}{10}\PY{p}{)}\PY{p}{,} \PY{n}{dpi}\PY{o}{=}\PY{l+m+mi}{80}\PY{p}{)}
         \PY{n}{plt}\PY{o}{.}\PY{n}{imshow}\PY{p}{(}\PY{n}{img\PYZus{}rubic}\PY{p}{[}\PY{l+m+mi}{1}\PY{p}{]}\PY{p}{,} \PY{n}{cmap}\PY{o}{=}\PY{l+s+s1}{\PYZsq{}}\PY{l+s+s1}{gray}\PY{l+s+s1}{\PYZsq{}}\PY{p}{)}
         \PY{n}{plt}\PY{o}{.}\PY{n}{quiver}\PY{p}{(}\PY{n}{grid}\PY{p}{[}\PY{l+m+mi}{0}\PY{p}{]}\PY{p}{,} \PY{n}{grid}\PY{p}{[}\PY{l+m+mi}{1}\PY{p}{]}\PY{p}{,} \PY{n}{flow}\PY{p}{[}\PY{p}{:}\PY{p}{,}\PY{p}{:}\PY{p}{,}\PY{l+m+mi}{1}\PY{p}{]}\PY{p}{,} \PY{n}{flow}\PY{p}{[}\PY{p}{:}\PY{p}{,}\PY{p}{:}\PY{p}{,}\PY{l+m+mi}{0}\PY{p}{]}\PY{p}{,} \PY{n}{color}\PY{o}{=}\PY{l+s+s1}{\PYZsq{}}\PY{l+s+s1}{lime}\PY{l+s+s1}{\PYZsq{}}\PY{p}{,} \PY{n}{angles}\PY{o}{=}\PY{l+s+s2}{\PYZdq{}}\PY{l+s+s2}{xy}\PY{l+s+s2}{\PYZdq{}}\PY{p}{,}\PY{n}{scale}\PY{o}{=}\PY{l+m+mi}{25}\PY{p}{)}
         \PY{n}{plt}\PY{o}{.}\PY{n}{show}\PY{p}{(}\PY{p}{)}
\end{Verbatim}


    \begin{center}
    \adjustimage{max size={0.9\linewidth}{0.9\paperheight}}{output_24_0.png}
    \end{center}
    { \hspace*{\fill} \\}
    
    \begin{Verbatim}[commandchars=\\\{\}]
{\color{incolor}In [{\color{incolor}17}]:} \PY{n}{grid}\PY{p}{,} \PY{n}{flow} \PY{o}{=} \PY{n}{lucas\PYZus{}kanade\PYZus{}optimzed}\PY{p}{(}\PY{n}{img\PYZus{}sphere}\PY{p}{[}\PY{l+m+mi}{0}\PY{p}{]}\PY{o}{/}\PY{l+m+mi}{255}\PY{p}{,} \PY{n}{img\PYZus{}sphere}\PY{p}{[}\PY{l+m+mi}{1}\PY{p}{]}\PY{o}{/}\PY{l+m+mi}{255}\PY{p}{,} \PY{l+m+mi}{8}\PY{p}{,} \PY{p}{(}\PY{l+m+mi}{40}\PY{p}{,} \PY{l+m+mi}{40}\PY{p}{)}\PY{p}{,} \PY{k+kc}{True}\PY{p}{)}
         \PY{n}{fig} \PY{o}{=} \PY{n}{plt}\PY{o}{.}\PY{n}{figure}\PY{p}{(}\PY{n}{figsize}\PY{o}{=}\PY{p}{(}\PY{l+m+mi}{10}\PY{p}{,} \PY{l+m+mi}{10}\PY{p}{)}\PY{p}{,} \PY{n}{dpi}\PY{o}{=}\PY{l+m+mi}{80}\PY{p}{)}
         \PY{n}{plt}\PY{o}{.}\PY{n}{imshow}\PY{p}{(}\PY{n}{img\PYZus{}sphere}\PY{p}{[}\PY{l+m+mi}{1}\PY{p}{]}\PY{p}{,} \PY{n}{cmap}\PY{o}{=}\PY{l+s+s1}{\PYZsq{}}\PY{l+s+s1}{gray}\PY{l+s+s1}{\PYZsq{}}\PY{p}{)}
         \PY{n}{plt}\PY{o}{.}\PY{n}{quiver}\PY{p}{(}\PY{n}{grid}\PY{p}{[}\PY{l+m+mi}{0}\PY{p}{]}\PY{p}{,} \PY{n}{grid}\PY{p}{[}\PY{l+m+mi}{1}\PY{p}{]}\PY{p}{,} \PY{n}{flow}\PY{p}{[}\PY{p}{:}\PY{p}{,}\PY{p}{:}\PY{p}{,}\PY{l+m+mi}{1}\PY{p}{]}\PY{p}{,} \PY{n}{flow}\PY{p}{[}\PY{p}{:}\PY{p}{,}\PY{p}{:}\PY{p}{,}\PY{l+m+mi}{0}\PY{p}{]}\PY{p}{,} \PY{n}{color}\PY{o}{=}\PY{l+s+s1}{\PYZsq{}}\PY{l+s+s1}{lime}\PY{l+s+s1}{\PYZsq{}}\PY{p}{,} \PY{n}{angles}\PY{o}{=}\PY{l+s+s2}{\PYZdq{}}\PY{l+s+s2}{xy}\PY{l+s+s2}{\PYZdq{}}\PY{p}{,}\PY{n}{scale}\PY{o}{=}\PY{l+m+mi}{50}\PY{p}{)}
         \PY{n}{plt}\PY{o}{.}\PY{n}{show}\PY{p}{(}\PY{p}{)}
\end{Verbatim}


    \begin{center}
    \adjustimage{max size={0.9\linewidth}{0.9\paperheight}}{output_25_0.png}
    \end{center}
    { \hspace*{\fill} \\}
    
    \hypertarget{conclusion}{%
\subsection{Conclusion}\label{conclusion}}

    The Lucas-Kanade Method provides a much more accurate representation of
the motion flow by using the gradients of the image to calculate the
motion vectors. This is more robust due to the fact that template
matching is vulnerable to noise.\\
In addition, much of the motion capture in the images are not
translations in the plane of the image. For instance, in the sphere
example, the left side of the sphere is moving towards the viewer and
the right side is moving away. This sort of motion is not a translation,
and is not captured well by template matching.


    % Add a bibliography block to the postdoc
    
    
    
    \end{document}
